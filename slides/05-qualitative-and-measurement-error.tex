% Options for packages loaded elsewhere
\PassOptionsToPackage{unicode}{hyperref}
\PassOptionsToPackage{hyphens}{url}
\PassOptionsToPackage{dvipsnames,svgnames,x11names}{xcolor}
%
\documentclass[
  ignorenonframetext,
]{beamer}
\usepackage{pgfpages}
\setbeamertemplate{caption}[numbered]
\setbeamertemplate{caption label separator}{: }
\setbeamercolor{caption name}{fg=normal text.fg}
\beamertemplatenavigationsymbolsempty
% Prevent slide breaks in the middle of a paragraph
\widowpenalties 1 10000
\raggedbottom
\setbeamertemplate{part page}{
  \centering
  \begin{beamercolorbox}[sep=16pt,center]{part title}
    \usebeamerfont{part title}\insertpart\par
  \end{beamercolorbox}
}
\setbeamertemplate{section page}{
  \centering
  \begin{beamercolorbox}[sep=12pt,center]{part title}
    \usebeamerfont{section title}\insertsection\par
  \end{beamercolorbox}
}
\setbeamertemplate{subsection page}{
  \centering
  \begin{beamercolorbox}[sep=8pt,center]{part title}
    \usebeamerfont{subsection title}\insertsubsection\par
  \end{beamercolorbox}
}
\AtBeginPart{
  \frame{\partpage}
}
\AtBeginSection{
  \ifbibliography
  \else
    \frame{\sectionpage}
  \fi
}
\AtBeginSubsection{
  \frame{\subsectionpage}
}
\usepackage{amsmath,amssymb}
\usepackage{iftex}
\ifPDFTeX
  \usepackage[T1]{fontenc}
  \usepackage[utf8]{inputenc}
  \usepackage{textcomp} % provide euro and other symbols
\else % if luatex or xetex
  \usepackage{unicode-math} % this also loads fontspec
  \defaultfontfeatures{Scale=MatchLowercase}
  \defaultfontfeatures[\rmfamily]{Ligatures=TeX,Scale=1}
\fi
\usepackage{lmodern}
\ifPDFTeX\else
  % xetex/luatex font selection
\fi
% Use upquote if available, for straight quotes in verbatim environments
\IfFileExists{upquote.sty}{\usepackage{upquote}}{}
\IfFileExists{microtype.sty}{% use microtype if available
  \usepackage[]{microtype}
  \UseMicrotypeSet[protrusion]{basicmath} % disable protrusion for tt fonts
}{}
\makeatletter
\@ifundefined{KOMAClassName}{% if non-KOMA class
  \IfFileExists{parskip.sty}{%
    \usepackage{parskip}
  }{% else
    \setlength{\parindent}{0pt}
    \setlength{\parskip}{6pt plus 2pt minus 1pt}}
}{% if KOMA class
  \KOMAoptions{parskip=half}}
\makeatother
\usepackage{xcolor}
\newif\ifbibliography
\usepackage{longtable,booktabs,array}
\usepackage{calc} % for calculating minipage widths
\usepackage{caption}
% Make caption package work with longtable
\makeatletter
\def\fnum@table{\tablename~\thetable}
\makeatother
\setlength{\emergencystretch}{3em} % prevent overfull lines
\providecommand{\tightlist}{%
  \setlength{\itemsep}{0pt}\setlength{\parskip}{0pt}}
\setcounter{secnumdepth}{5}
\usepackage{/home/jb/R/x86_64-pc-linux-gnu-library/4.3/quack/rmarkdown/templates/presentation/resources/beamerthemeAustin}
\usepackage{/home/jb/R/x86_64-pc-linux-gnu-library/4.3/quack/rmarkdown/templates/presentation/resources/beamercolorthemelonghorn}
\newcommand{\setsep}{\setlength{\itemsep}{3pt}}
\newcommand{\setskip}{\setlength{\parskip}{3pt}}
\renewcommand{\tightlist}{\setsep\setskip}
\newcommand{\expectation}[1]{\ensuremath{\mathbb{E}\left[#1\right]}}
\newcommand{\pd}[2][]{\ensuremath{\frac{\partial{#1}}{\partial{#2}}}}
\DeclareMathOperator{\Var}{Var}
\DeclareMathOperator{\Cov}{Cov}
\newcommand{\variance}[1]{\ensuremath{\Var\left[#1\right]}}
\DeclareMathOperator{\standarddeviation}{sd}
\DeclareMathOperator{\standarderror}{se}
\usepackage{siunitx}
\ifLuaTeX
  \usepackage{selnolig}  % disable illegal ligatures
\fi
\IfFileExists{bookmark.sty}{\usepackage{bookmark}}{\usepackage{hyperref}}
\IfFileExists{xurl.sty}{\usepackage{xurl}}{} % add URL line breaks if available
\urlstyle{same}
\hypersetup{
  pdftitle={Statistical Analysis in Political Science II:Qualitative variables and measurement error},
  pdfauthor={JBrandon Duck-Mayr},
  colorlinks=true,
  linkcolor={Maroon},
  filecolor={Maroon},
  citecolor={Blue},
  urlcolor={Blue},
  pdfcreator={LaTeX via pandoc}}

\title{Statistical Analysis in Political Science II:\newline Qualitative variables and measurement error}
\author{JBrandon Duck-Mayr}
\date{February 28, 2024}

\begin{document}
\frame{\titlepage}

\begin{frame}{Binary variables}
\protect\hypertarget{binary-variables}{}
\pause

\begin{itemize}[<+->]
\tightlist
\item
  Also called dummy variables or dichotomous variables
\item
  Encode a specific type of qualitative variable, one characteristic that can take only one of two values
\item
  Examples (some better than others): treatment indicator, whether it's an election year, gender
\end{itemize}

\pause

With dummy variable \(D\) and other predictors \(x_2, \dots, x_k\),
in the model
\[
y = \beta_0 + \beta_1 D + \beta_2 x_2 + \dots + \beta_k x_k + \varepsilon,
\]
the coefficient \(\beta_1\) on \(D\) represents an \textbf{intercept shift} when \(D = 1\)
\end{frame}

\begin{frame}{Categorical variables}
\protect\hypertarget{categorical-variables}{}
\pause

\begin{itemize}[<+->]
\tightlist
\item
  What about characteristics with more than one value, like party identification?
\item
  Omit one category and create dummy variables for each other category
\item
  Then we interpret the dummy coefficients as an intercept difference between that group and the omitted group
\item
  Does it matter if the information is ordinal, like education?
\end{itemize}
\end{frame}

\begin{frame}{Dummy:continuous interactions}
\protect\hypertarget{dummycontinuous-interactions}{}
\pause

\begin{itemize}[<+->]
\tightlist
\item
  (note we will cover this in more detail next time)
\item
  In the model \[ y = \beta_0 + \beta_1 D + \beta_2 x + \beta_3 D x + \varepsilon, \] \(\beta_1\) represents an intercept shift when \(D = 1\), but we also allow for \textbf{different slopes}
\item
  Then \(\beta_2\) is the slope on \(x\) when \(D = 0\) and \(\beta_2 + \beta_3\) is the slope on \(x\) when \(D = 1\)
\end{itemize}
\end{frame}

\begin{frame}{Testing differences across groups}
\protect\hypertarget{testing-differences-across-groups}{}
\pause

\begin{itemize}[<+->]
\tightlist
\item
  Suppose we want to let all the slopes differ across groups,
  \[ y = \beta_0 + \beta_{g,1} x_1 + \beta_{g, 2} x_2 + \dots + \beta_{g, k} x_k + \varepsilon \]
\item
  We can do an \(F\)-test with \(k + 1\) restrictions
\item
  We can also run separate regressions

  \begin{itemize}[<+->]
  \tightlist
  \item
    Let \(\text{SSR}_1\) be the SSR for a regression for the first group
  \item
    Let \(\text{SSR}_2\) be the SSR for a regression for the second
  \item
    Let \(\text{SSR}_p\) be the SSR for a pooled regression
  \item
    Then we compute the \(F\)-statistic \[ F = \frac{\text{SSR}_p - (\text{SSR}_1 + \text{SSR}_2)}{\text{SSR}_1 + \text{SSR}_2} \frac{n - 2(k+1)}{k + 1}, \] with \(k\) and \(n_1 + n_2 - 2k\) d.o.f.
  \item
    This is called a Chow test
  \end{itemize}
\end{itemize}
\end{frame}

\begin{frame}{Measurement error in the outcome}
\protect\hypertarget{measurement-error-in-the-outcome}{}
\pause

\begin{itemize}[<+->]
\tightlist
\item
  Say we want to estimate the equation \(y^\ast = \beta_0 + \beta_1 x_1 + \dots + \beta_k x_k + \varepsilon,\) but only observe \(y = y^\ast + e_0\)
\item
  Then we estimate \(y = \beta_0 + \beta_1 x_1 + \dots + \beta_k x_k + \varepsilon^\ast\), where \(\varepsilon^\ast = \varepsilon + e_0\)
\item
  This is valid when \(e_0\) is unrelated to the predictors

  \begin{itemize}[<+->]
  \tightlist
  \item
    Notice the effect on estimator variance: \(\ensuremath{\mathop{\mathrm{Var}}\left[\varepsilon^\ast\right]} = \ensuremath{\mathop{\mathrm{Var}}\left[\varepsilon + e_0\right]} = \sigma_\varepsilon^2 + \sigma_0^2 > \sigma_\varepsilon^2\)
  \end{itemize}
\item
  This is \textbf{not} valid if \(e_0\) is related to the predictors
\end{itemize}
\end{frame}

\begin{frame}{Measurement error in the predictors}
\protect\hypertarget{measurement-error-in-the-predictors}{}
\pause

\begin{itemize}[<+->]
\tightlist
\item
  Say we want to estimate the equation \(y = \beta_0 + \beta_1 x_1^\ast + \dots + \beta_k x_k + \varepsilon,\) but only observe \(x_1 = x_1^\ast + e_1\)
\item
  Two cases:
\item
  \(\mathop{\mathrm{Cov}}\left(x_1, e_1\right) = 0\)

  \begin{itemize}[<+->]
  \tightlist
  \item
    Estimate \(y = \beta_0 + \beta_1 x_1 + \dots + \beta_k x_k + \left(\varepsilon - \beta_1 e_1\right)\)
  \item
    Still unbiased
  \item
    Variance increases: \(\ensuremath{\mathop{\mathrm{Var}}\left[\varepsilon - \beta_1 e_1\right]} = \sigma_\varepsilon^2 + \beta_1 \sigma_{e_1}^2\)
  \end{itemize}
\item
  \(\mathop{\mathrm{Cov}}\left(x_1^\ast, e_1\right) = 0\)

  \begin{itemize}[<+->]
  \tightlist
  \item
    classical errors-in-variables (CEV) assumption
  \item
    then \(e_1\) and \(x_1\) are correlated
  \item
    covariance between \(x_1\) and \(\left(\varepsilon - \beta_1 e_1\right)\) is
    \[ \mathop{\mathrm{Cov}}\left(x_1, \varepsilon - \beta_1 e_1\right) = -\beta_1 \mathop{\mathrm{Cov}}\left(x_1, e_1\right) = \beta_1\sigma_{e_1}^2\]
  \end{itemize}
\end{itemize}
\end{frame}

\begin{frame}{Measurement error in the predictors}
\protect\hypertarget{measurement-error-in-the-predictors-1}{}
Then in the two variable case,

\begin{align*}
\text{plim}\left(\hat{\beta}_1\right)
& = \beta_1 + \frac{\mathop{\mathrm{Cov}}\left(x_1, \varepsilon - \beta_1 e_1\right)}{\ensuremath{\mathop{\mathrm{Var}}\left[x_1\right]}} \\
& = \beta_1 - \frac{\beta_1\sigma_{e_1}^2}{\sigma_{x_1^\ast}^2 + \sigma_{e_1}^2} \\
& = \beta_1 \left( \frac{\sigma_{x_1^\ast}^2}{\sigma_{x_1^\ast}^2 + \sigma_{e_1}^2} \right)
\end{align*}

\pause

For \(k\) variables,

\[
\text{plim}\left(\hat{\beta}_1\right)
=
\beta_1 \left( \frac{\sigma_{r_1^\ast}^2}{\sigma_{r_1^\ast}^2 + \sigma_{e_1}^2} \right)
\]
\end{frame}

\begin{frame}{Missing data}
\protect\hypertarget{missing-data}{}
\pause

\begin{itemize}[<+->]
\tightlist
\item
  Different types of missingness:

  \begin{itemize}[<+->]
  \tightlist
  \item
    \textcolor<7-8>{BurntOrange}{MCAR (missing completely at random): missingness unrelated to observed \textit{and} unobserved data}
  \item
    \textcolor<8>{BurntOrange}{MAR (missing at random): missingness only unrelated to unobserved data}
  \item
    MNAR (missing not at random): missingess systematically related even to unobserved data
  \end{itemize}
\item
  Strategies to deal with missingness:

  \begin{itemize}[<+->]
  \tightlist
  \item
    listwise deletion
  \item
    multiple imputation
  \end{itemize}
\item
  Missingness in predictors vs outcomes?
\end{itemize}
\end{frame}

\end{document}
