% Options for packages loaded elsewhere
\PassOptionsToPackage{unicode}{hyperref}
\PassOptionsToPackage{hyphens}{url}
\PassOptionsToPackage{dvipsnames,svgnames,x11names}{xcolor}
%
\documentclass[
  ignorenonframetext,
]{beamer}
\usepackage{pgfpages}
\setbeamertemplate{caption}[numbered]
\setbeamertemplate{caption label separator}{: }
\setbeamercolor{caption name}{fg=normal text.fg}
\beamertemplatenavigationsymbolsempty
% Prevent slide breaks in the middle of a paragraph
\widowpenalties 1 10000
\raggedbottom
\setbeamertemplate{part page}{
  \centering
  \begin{beamercolorbox}[sep=16pt,center]{part title}
    \usebeamerfont{part title}\insertpart\par
  \end{beamercolorbox}
}
\setbeamertemplate{section page}{
  \centering
  \begin{beamercolorbox}[sep=12pt,center]{part title}
    \usebeamerfont{section title}\insertsection\par
  \end{beamercolorbox}
}
\setbeamertemplate{subsection page}{
  \centering
  \begin{beamercolorbox}[sep=8pt,center]{part title}
    \usebeamerfont{subsection title}\insertsubsection\par
  \end{beamercolorbox}
}
\AtBeginPart{
  \frame{\partpage}
}
\AtBeginSection{
  \ifbibliography
  \else
    \frame{\sectionpage}
  \fi
}
\AtBeginSubsection{
  \frame{\subsectionpage}
}
\usepackage{amsmath,amssymb}
\usepackage{iftex}
\ifPDFTeX
  \usepackage[T1]{fontenc}
  \usepackage[utf8]{inputenc}
  \usepackage{textcomp} % provide euro and other symbols
\else % if luatex or xetex
  \usepackage{unicode-math} % this also loads fontspec
  \defaultfontfeatures{Scale=MatchLowercase}
  \defaultfontfeatures[\rmfamily]{Ligatures=TeX,Scale=1}
\fi
\usepackage{lmodern}
\ifPDFTeX\else
  % xetex/luatex font selection
\fi
% Use upquote if available, for straight quotes in verbatim environments
\IfFileExists{upquote.sty}{\usepackage{upquote}}{}
\IfFileExists{microtype.sty}{% use microtype if available
  \usepackage[]{microtype}
  \UseMicrotypeSet[protrusion]{basicmath} % disable protrusion for tt fonts
}{}
\makeatletter
\@ifundefined{KOMAClassName}{% if non-KOMA class
  \IfFileExists{parskip.sty}{%
    \usepackage{parskip}
  }{% else
    \setlength{\parindent}{0pt}
    \setlength{\parskip}{6pt plus 2pt minus 1pt}}
}{% if KOMA class
  \KOMAoptions{parskip=half}}
\makeatother
\usepackage{xcolor}
\newif\ifbibliography
\usepackage{color}
\usepackage{fancyvrb}
\newcommand{\VerbBar}{|}
\newcommand{\VERB}{\Verb[commandchars=\\\{\}]}
\DefineVerbatimEnvironment{Highlighting}{Verbatim}{commandchars=\\\{\}}
% Add ',fontsize=\small' for more characters per line
\usepackage{framed}
\definecolor{shadecolor}{RGB}{248,248,248}
\newenvironment{Shaded}{\begin{snugshade}}{\end{snugshade}}
\newcommand{\AlertTok}[1]{\textcolor[rgb]{0.94,0.16,0.16}{#1}}
\newcommand{\AnnotationTok}[1]{\textcolor[rgb]{0.56,0.35,0.01}{\textbf{\textit{#1}}}}
\newcommand{\AttributeTok}[1]{\textcolor[rgb]{0.13,0.29,0.53}{#1}}
\newcommand{\BaseNTok}[1]{\textcolor[rgb]{0.00,0.00,0.81}{#1}}
\newcommand{\BuiltInTok}[1]{#1}
\newcommand{\CharTok}[1]{\textcolor[rgb]{0.31,0.60,0.02}{#1}}
\newcommand{\CommentTok}[1]{\textcolor[rgb]{0.56,0.35,0.01}{\textit{#1}}}
\newcommand{\CommentVarTok}[1]{\textcolor[rgb]{0.56,0.35,0.01}{\textbf{\textit{#1}}}}
\newcommand{\ConstantTok}[1]{\textcolor[rgb]{0.56,0.35,0.01}{#1}}
\newcommand{\ControlFlowTok}[1]{\textcolor[rgb]{0.13,0.29,0.53}{\textbf{#1}}}
\newcommand{\DataTypeTok}[1]{\textcolor[rgb]{0.13,0.29,0.53}{#1}}
\newcommand{\DecValTok}[1]{\textcolor[rgb]{0.00,0.00,0.81}{#1}}
\newcommand{\DocumentationTok}[1]{\textcolor[rgb]{0.56,0.35,0.01}{\textbf{\textit{#1}}}}
\newcommand{\ErrorTok}[1]{\textcolor[rgb]{0.64,0.00,0.00}{\textbf{#1}}}
\newcommand{\ExtensionTok}[1]{#1}
\newcommand{\FloatTok}[1]{\textcolor[rgb]{0.00,0.00,0.81}{#1}}
\newcommand{\FunctionTok}[1]{\textcolor[rgb]{0.13,0.29,0.53}{\textbf{#1}}}
\newcommand{\ImportTok}[1]{#1}
\newcommand{\InformationTok}[1]{\textcolor[rgb]{0.56,0.35,0.01}{\textbf{\textit{#1}}}}
\newcommand{\KeywordTok}[1]{\textcolor[rgb]{0.13,0.29,0.53}{\textbf{#1}}}
\newcommand{\NormalTok}[1]{#1}
\newcommand{\OperatorTok}[1]{\textcolor[rgb]{0.81,0.36,0.00}{\textbf{#1}}}
\newcommand{\OtherTok}[1]{\textcolor[rgb]{0.56,0.35,0.01}{#1}}
\newcommand{\PreprocessorTok}[1]{\textcolor[rgb]{0.56,0.35,0.01}{\textit{#1}}}
\newcommand{\RegionMarkerTok}[1]{#1}
\newcommand{\SpecialCharTok}[1]{\textcolor[rgb]{0.81,0.36,0.00}{\textbf{#1}}}
\newcommand{\SpecialStringTok}[1]{\textcolor[rgb]{0.31,0.60,0.02}{#1}}
\newcommand{\StringTok}[1]{\textcolor[rgb]{0.31,0.60,0.02}{#1}}
\newcommand{\VariableTok}[1]{\textcolor[rgb]{0.00,0.00,0.00}{#1}}
\newcommand{\VerbatimStringTok}[1]{\textcolor[rgb]{0.31,0.60,0.02}{#1}}
\newcommand{\WarningTok}[1]{\textcolor[rgb]{0.56,0.35,0.01}{\textbf{\textit{#1}}}}
\usepackage{longtable,booktabs,array}
\usepackage{calc} % for calculating minipage widths
\usepackage{caption}
% Make caption package work with longtable
\makeatletter
\def\fnum@table{\tablename~\thetable}
\makeatother
\setlength{\emergencystretch}{3em} % prevent overfull lines
\providecommand{\tightlist}{%
  \setlength{\itemsep}{0pt}\setlength{\parskip}{0pt}}
\setcounter{secnumdepth}{5}
\usepackage{/home/jb/R/x86_64-pc-linux-gnu-library/4.4/quack/rmarkdown/templates/presentation/resources/beamerthemeAustin}
\usepackage{/home/jb/R/x86_64-pc-linux-gnu-library/4.4/quack/rmarkdown/templates/presentation/resources/beamercolorthemelonghorn}
\newcommand{\setsep}{\setlength{\itemsep}{3pt}}
\newcommand{\setskip}{\setlength{\parskip}{3pt}}
\renewcommand{\tightlist}{\setsep\setskip}
\newcommand{\expectation}[1]{\ensuremath{\mathbb{E}\left[#1\right]}}
\newcommand{\pd}[2][]{\ensuremath{\frac{\partial{#1}}{\partial{#2}}}}
\DeclareMathOperator{\Var}{Var}
\DeclareMathOperator{\Cov}{Cov}
\newcommand{\variance}[1]{\ensuremath{\Var\left[#1\right]}}
\newcommand{\covariance}[1]{\ensuremath{\Cov\left[#1\right]}}
\DeclareMathOperator{\standarddeviation}{sd}
\DeclareMathOperator{\standarderror}{se}
\usepackage{siunitx}
\ifLuaTeX
  \usepackage{selnolig}  % disable illegal ligatures
\fi
\usepackage{bookmark}
\IfFileExists{xurl.sty}{\usepackage{xurl}}{} % add URL line breaks if available
\urlstyle{same}
\hypersetup{
  pdftitle={Statistical Analysis in Political Science II:Interaction Terms and Diagnostics},
  pdfauthor={JBrandon Duck-Mayr},
  colorlinks=true,
  linkcolor={Maroon},
  filecolor={Maroon},
  citecolor={Blue},
  urlcolor={Blue},
  pdfcreator={LaTeX via pandoc}}

\title{Statistical Analysis in Political Science II:\newline Interaction Terms and Diagnostics}
\author{JBrandon Duck-Mayr}
\date{February 26, 2025}

\begin{document}
\frame{\titlepage}

\begin{frame}{Background: Marginal effects}
\phantomsection\label{background-marginal-effects}
\pause

\begin{itemize}[<+->]
\tightlist
\item
  The marginal effect of a predictor is how the outcome changes as the predictor changes
\item
  So far this has just been the regression coefficient
\item
  Recall: derivatives tell you rate of change; the partial derivative with respect to the predictor of interest tells you the marginal effect
\item
  Simple example: In the equation \(y = \beta_0 + \beta_1 x_1 + \varepsilon\), the marginal effect of \(x_1\) is \(\ensuremath{\frac{\partial{y}}{\partial{x_1}}} = \beta_1\)
\end{itemize}
\end{frame}

\begin{frame}{Motivating interaction terms: Conditional effects}
\phantomsection\label{motivating-interaction-terms-conditional-effects}
\pause

\begin{itemize}[<+->]
\tightlist
\item
  Social scientists often have conditional hypotheses
\item
  Example: Hypothesizing that the effect of a Supreme Court justice's ideology on their interruptions in oral argument is positive when the Chief Justice is conservative and negative when the Chief is liberal
\item
  Example: Hypothesizing that the effect of campaign ad buys on vote choice decreases with name recognition
\item
  In a normal linear model, the marginal effect of one variable cannot depend on the value of another

  \begin{itemize}[<+->]
  \tightlist
  \item
    If \(y = \beta_0 + \beta_1 x_1 + \dots + \beta_k x_k\), then \(\ensuremath{\frac{\partial{y}}{\partial{x_j}}} = \beta_j\)\ldots{} there are no \(X\)'s there!
  \end{itemize}
\item
  So we include interaction terms
\end{itemize}
\end{frame}

\begin{frame}{What do they \emph{mean}?}
\phantomsection\label{what-do-they-mean}
\pause

\begin{itemize}[<+->]
\tightlist
\item
  In a model \(y = \beta_0 + \beta_1 x_1 + \beta_2 x_2 + \beta_3 x_1 x_2\),

  \begin{itemize}[<+->]
  \tightlist
  \item
    If \(x_2\) is dichotomous, we are saying the effect of \(x_1\) depends on whether condition \(x_2\) is present or not
  \item
    If \(x_2\) is continuous, we are saying the effect of \(x_1\) can be different (magnitude and/or sign) as \(x_2\) increases or decreases
  \end{itemize}
\item
  Then \(\ensuremath{\frac{\partial{y}}{\partial{x_1}}} = \beta_1 + \beta_3 x_2\)---success! \(x_2\) is in the formula for the marginal effect of \(x_1\)

  \begin{itemize}[<+->]
  \tightlist
  \item
    So how much does an increase in \(x_1\) affect \(y\)? It depends! (on \(x_2\))
  \end{itemize}
\item
  \textbf{Careful}: Not only is \(\ensuremath{\frac{\partial{y}}{\partial{x_1}}}\) not equal to \(\beta_1\), or even \(\beta_1 + \beta_3\)\onslide<8->{, but:}

  \begin{itemize}[<+->]
  \tightlist
  \item
    \(\mathop{\mathrm{sd}}\left( \ensuremath{\frac{\partial{y}}{\partial{x_1}}} \right) \neq \mathop{\mathrm{sd}}\left( \beta_1 \right)\)
  \item
    \(\mathop{\mathrm{sd}}\left( \ensuremath{\frac{\partial{y}}{\partial{x_1}}} \right) \neq \mathop{\mathrm{sd}}\left( \beta_3 \right)\)
  \item
    \(\mathop{\mathrm{sd}}\left( \ensuremath{\frac{\partial{y}}{\partial{x_1}}} \right) \neq \mathop{\mathrm{sd}}\left( \beta_1 \right) + \mathop{\mathrm{sd}}\left( \beta_3 \right)\)
  \end{itemize}
\end{itemize}
\end{frame}

\begin{frame}{Interaction terms and hypothesis tests}
\phantomsection\label{interaction-terms-and-hypothesis-tests}
\pause

\begin{itemize}[<+->]
\tightlist
\item
  Recall:

  \begin{itemize}[<+->]
  \tightlist
  \item
    \(\ensuremath{\mathop{\mathrm{Var}}\left[X + Y\right]} = \ensuremath{\mathop{\mathrm{Var}}\left[X\right]} + \ensuremath{\mathop{\mathrm{Var}}\left[Y\right]} + 2\ensuremath{\mathop{\mathrm{Cov}}\left[X, Y\right]}\)
  \item
    \(\ensuremath{\mathop{\mathrm{Var}}\left[c X\right]} = c^2 \ensuremath{\mathop{\mathrm{Var}}\left[X\right]}\)
  \end{itemize}
\item
  So! \(\mathop{\mathrm{se}}\left(\hat{\beta}_1 + \hat{\beta}_3 x_2\right) = \sqrt{\ensuremath{\mathop{\mathrm{Var}}\left[\hat{\beta}_1\right]} + x_2^2 \ensuremath{\mathop{\mathrm{Var}}\left[\hat{\beta}_3\right]} + 2 x_2 \ensuremath{\mathop{\mathrm{Cov}}\left[\hat{\beta}_1, \hat{\beta}_3\right]}}\)

  \begin{itemize}[<+->]
  \tightlist
  \item
    Whether or not \(x_1\) has a statistically significant effect on \(y\) depends on the value of \(x_2\); for some values of \(x_2\) the effect is significant, for others it isn't
  \end{itemize}
\end{itemize}
\end{frame}

\begin{frame}[fragile]{Dichotomous example: Simulation}
\phantomsection\label{dichotomous-example-simulation}
\pause

First simulate some data

\begin{Shaded}
\begin{Highlighting}[]
\FunctionTok{set.seed}\NormalTok{(}\DecValTok{42}\NormalTok{)}
\NormalTok{n  }\OtherTok{=} \DecValTok{100}
\NormalTok{x1 }\OtherTok{=} \FunctionTok{rnorm}\NormalTok{(n)}
\NormalTok{x2 }\OtherTok{=} \FunctionTok{sample}\NormalTok{(}\DecValTok{0}\SpecialCharTok{:}\DecValTok{1}\NormalTok{, }\AttributeTok{size =}\NormalTok{ n, }\AttributeTok{replace =} \ConstantTok{TRUE}\NormalTok{)}
\NormalTok{X  }\OtherTok{=} \FunctionTok{cbind}\NormalTok{(}\DecValTok{1}\NormalTok{, x1, x2, x1}\SpecialCharTok{*}\NormalTok{x2)}
\NormalTok{b  }\OtherTok{=} \FunctionTok{c}\NormalTok{(}\DecValTok{1}\NormalTok{, }\DecValTok{1}\NormalTok{, }\DecValTok{1}\NormalTok{, }\SpecialCharTok{{-}}\DecValTok{2}\NormalTok{)}
\NormalTok{e  }\OtherTok{=} \FunctionTok{rnorm}\NormalTok{(n)}
\NormalTok{y  }\OtherTok{=} \FunctionTok{c}\NormalTok{(X }\SpecialCharTok{\%*\%}\NormalTok{ b }\SpecialCharTok{+}\NormalTok{ e)}
\end{Highlighting}
\end{Shaded}
\end{frame}

\begin{frame}[fragile]{Dichotomous example: Calculation}
\phantomsection\label{dichotomous-example-calculation}
Now fit the model

\begin{Shaded}
\begin{Highlighting}[]
\NormalTok{m }\OtherTok{=} \FunctionTok{lm}\NormalTok{(y }\SpecialCharTok{\textasciitilde{}}\NormalTok{ x1}\SpecialCharTok{*}\NormalTok{x2)}
\end{Highlighting}
\end{Shaded}

\pause

And calculate the marginal effect of \texttt{x1} when \texttt{x2} is 0 or 1

\begin{Shaded}
\begin{Highlighting}[]
\NormalTok{V }\OtherTok{=} \FunctionTok{vcov}\NormalTok{(m, }\AttributeTok{complete =} \ConstantTok{TRUE}\NormalTok{) }\CommentTok{\# \textbackslash{}Var(\textbackslash{}hat\{\textbackslash{}beta\})}
\NormalTok{ME1 }\OtherTok{=} \FunctionTok{coef}\NormalTok{(m)[}\DecValTok{2}\NormalTok{] }\SpecialCharTok{+} \FunctionTok{coef}\NormalTok{(m)[}\DecValTok{4}\NormalTok{]}
\NormalTok{SE1 }\OtherTok{=} \FunctionTok{sqrt}\NormalTok{(V[}\DecValTok{2}\NormalTok{, }\DecValTok{2}\NormalTok{] }\SpecialCharTok{+}\NormalTok{ V[}\DecValTok{4}\NormalTok{, }\DecValTok{4}\NormalTok{] }\SpecialCharTok{+} \DecValTok{2}\SpecialCharTok{*}\NormalTok{V[}\DecValTok{2}\NormalTok{, }\DecValTok{4}\NormalTok{])}
\FunctionTok{sprintf}\NormalTok{(}\StringTok{"ME of x1 when x2=1 is \%0.3f (s.e. \%0.3f)"}\NormalTok{, ME1, SE1)}
\end{Highlighting}
\end{Shaded}

\begin{verbatim}
## [1] "ME of x1 when x2=1 is -0.975 (s.e. 0.131)"
\end{verbatim}

\begin{Shaded}
\begin{Highlighting}[]
\NormalTok{ME0 }\OtherTok{=} \FunctionTok{coef}\NormalTok{(m)[}\DecValTok{2}\NormalTok{]}
\NormalTok{SE0 }\OtherTok{=} \FunctionTok{sqrt}\NormalTok{(V[}\DecValTok{2}\NormalTok{, }\DecValTok{2}\NormalTok{])}
\FunctionTok{sprintf}\NormalTok{(}\StringTok{"ME of x1 when x2=0 is \%0.3f (s.e. \%0.3f)"}\NormalTok{, ME0, SE0)}
\end{Highlighting}
\end{Shaded}

\begin{verbatim}
## [1] "ME of x1 when x2=0 is 1.108 (s.e. 0.126)"
\end{verbatim}
\end{frame}

\begin{frame}[fragile]{Dichotomous example: comparison to \texttt{\{margins\}}}
\phantomsection\label{dichotomous-example-comparison-to-margins}
\begin{Shaded}
\begin{Highlighting}[]
\FunctionTok{library}\NormalTok{(margins)}
\NormalTok{MEs }\OtherTok{=} \FunctionTok{margins}\NormalTok{(m, }\AttributeTok{variables =} \StringTok{"x1"}\NormalTok{, }\AttributeTok{at =} \FunctionTok{list}\NormalTok{(}\AttributeTok{x2 =} \DecValTok{0}\SpecialCharTok{:}\DecValTok{1}\NormalTok{))}
\FunctionTok{print}\NormalTok{(}\FunctionTok{summary}\NormalTok{(MEs), }\AttributeTok{digits =} \DecValTok{3}\NormalTok{)}
\end{Highlighting}
\end{Shaded}

\begin{verbatim}
##  factor    x2    AME    SE      z     p  lower  upper
##      x1 0.000  1.108 0.126  8.788 0.000  0.861  1.355
##      x1 1.000 -0.975 0.131 -7.472 0.000 -1.231 -0.720
\end{verbatim}

\begin{Shaded}
\begin{Highlighting}[]
\DocumentationTok{\#\# Manual calculation:}
\NormalTok{E }\OtherTok{=} \FunctionTok{c}\NormalTok{(ME0,ME1); SE }\OtherTok{=} \FunctionTok{c}\NormalTok{(SE0,SE1); z}\OtherTok{=}\NormalTok{E}\SpecialCharTok{/}\NormalTok{SE; k}\OtherTok{=}\FunctionTok{qnorm}\NormalTok{(}\FloatTok{0.975}\NormalTok{)}\SpecialCharTok{*}\NormalTok{SE}
\FunctionTok{print}\NormalTok{(}\FunctionTok{data.frame}\NormalTok{(}
    \AttributeTok{factor =} \StringTok{"x1"}\NormalTok{, }\AttributeTok{x2 =} \FunctionTok{c}\NormalTok{(}\DecValTok{0}\NormalTok{, }\DecValTok{1}\NormalTok{), }\AttributeTok{AME =}\NormalTok{ E, }\AttributeTok{SE =}\NormalTok{ SE,}
    \AttributeTok{z =}\NormalTok{ z, }\AttributeTok{p =} \DecValTok{1}\SpecialCharTok{{-}}\FunctionTok{pnorm}\NormalTok{(}\FunctionTok{abs}\NormalTok{(z)), }\AttributeTok{lower =}\NormalTok{ E}\SpecialCharTok{{-}}\NormalTok{k, }\AttributeTok{upper =}\NormalTok{ E}\SpecialCharTok{+}\NormalTok{k}
\NormalTok{), }\AttributeTok{digits =} \DecValTok{3}\NormalTok{)}
\end{Highlighting}
\end{Shaded}

\begin{verbatim}
##   factor x2    AME    SE     z        p  lower upper
## 1     x1  0  1.108 0.126  8.79 0.00e+00  0.861  1.35
## 2     x1  1 -0.975 0.131 -7.47 3.94e-14 -1.231 -0.72
\end{verbatim}
\end{frame}

\begin{frame}{Dichotomous example}
\phantomsection\label{dichotomous-example}
\includegraphics[width=0.9\linewidth]{07-interactions-diagnostics_files/figure-beamer/unnamed-chunk-5-1}
\end{frame}

\begin{frame}[fragile]{Continuous example: Simulation}
\phantomsection\label{continuous-example-simulation}
\pause

First simulate some data

\begin{Shaded}
\begin{Highlighting}[]
\FunctionTok{set.seed}\NormalTok{(}\DecValTok{42}\NormalTok{)}
\NormalTok{n  }\OtherTok{=} \DecValTok{100}
\NormalTok{x1 }\OtherTok{=} \FunctionTok{rnorm}\NormalTok{(n)}
\NormalTok{x2 }\OtherTok{=} \FunctionTok{rnorm}\NormalTok{(n)}
\NormalTok{X  }\OtherTok{=} \FunctionTok{cbind}\NormalTok{(}\DecValTok{1}\NormalTok{, x1, x2, x1}\SpecialCharTok{*}\NormalTok{x2)}
\NormalTok{b  }\OtherTok{=} \FunctionTok{c}\NormalTok{(}\DecValTok{1}\NormalTok{, }\DecValTok{1}\NormalTok{, }\DecValTok{1}\NormalTok{, }\SpecialCharTok{{-}}\DecValTok{2}\NormalTok{)}
\NormalTok{e  }\OtherTok{=} \FunctionTok{rnorm}\NormalTok{(n)}
\NormalTok{y  }\OtherTok{=} \FunctionTok{c}\NormalTok{(X }\SpecialCharTok{\%*\%}\NormalTok{ b }\SpecialCharTok{+}\NormalTok{ e)}
\end{Highlighting}
\end{Shaded}
\end{frame}

\begin{frame}[fragile]{Continuous example: Calculation}
\phantomsection\label{continuous-example-calculation}
Now fit the model

\begin{Shaded}
\begin{Highlighting}[]
\NormalTok{m }\OtherTok{=} \FunctionTok{lm}\NormalTok{(y }\SpecialCharTok{\textasciitilde{}}\NormalTok{ x1}\SpecialCharTok{*}\NormalTok{x2); V }\OtherTok{=} \FunctionTok{vcov}\NormalTok{(m, }\AttributeTok{complete =} \ConstantTok{TRUE}\NormalTok{)}
\end{Highlighting}
\end{Shaded}

\pause

And calculate the marginal effect of \texttt{x1}

\begin{Shaded}
\begin{Highlighting}[]
\NormalTok{mx }\OtherTok{=} \FunctionTok{mean}\NormalTok{(x2)}
\NormalTok{ME }\OtherTok{=} \FunctionTok{coef}\NormalTok{(m)[}\DecValTok{2}\NormalTok{] }\SpecialCharTok{+} \FunctionTok{coef}\NormalTok{(m)[}\DecValTok{4}\NormalTok{] }\SpecialCharTok{*}\NormalTok{ mx}
\NormalTok{SE }\OtherTok{=} \FunctionTok{sqrt}\NormalTok{(V[}\DecValTok{2}\NormalTok{, }\DecValTok{2}\NormalTok{] }\SpecialCharTok{+}\NormalTok{ mx}\SpecialCharTok{\^{}}\DecValTok{2} \SpecialCharTok{*}\NormalTok{ V[}\DecValTok{4}\NormalTok{, }\DecValTok{4}\NormalTok{] }\SpecialCharTok{+} \DecValTok{2}\SpecialCharTok{*}\NormalTok{mx}\SpecialCharTok{*}\NormalTok{V[}\DecValTok{2}\NormalTok{, }\DecValTok{4}\NormalTok{])}
\NormalTok{msg }\OtherTok{=} \StringTok{"when x2=mean(x2) ME of x1 = \%0.3f (s.e. \%0.3f)"}
\FunctionTok{sprintf}\NormalTok{(msg, ME, SE)}
\end{Highlighting}
\end{Shaded}

\begin{verbatim}
## [1] "when x2=mean(x2) ME of x1 = 1.033 (s.e. 0.097)"
\end{verbatim}

\begin{Shaded}
\begin{Highlighting}[]
\NormalTok{MEs }\OtherTok{=} \FunctionTok{margins}\NormalTok{(m, }\AttributeTok{variables =} \StringTok{"x1"}\NormalTok{, }\AttributeTok{at =} \FunctionTok{list}\NormalTok{(}\AttributeTok{x2 =}\NormalTok{ mx))}
\FunctionTok{print}\NormalTok{(}\FunctionTok{summary}\NormalTok{(MEs), }\AttributeTok{digits =} \DecValTok{3}\NormalTok{)}
\end{Highlighting}
\end{Shaded}

\begin{verbatim}
##  factor     x2   AME    SE      z     p lower upper
##      x1 -0.087 1.033 0.097 10.638 0.000 0.843 1.223
\end{verbatim}
\end{frame}

\begin{frame}[fragile]{Continuous example}
\phantomsection\label{continuous-example}
\begin{Shaded}
\begin{Highlighting}[]
\NormalTok{margins}\SpecialCharTok{::}\FunctionTok{cplot}\NormalTok{(m, }\AttributeTok{x =} \StringTok{"x2"}\NormalTok{, }\AttributeTok{dx =} \StringTok{"x1"}\NormalTok{, }\AttributeTok{what =} \StringTok{"effect"}\NormalTok{)}
\FunctionTok{abline}\NormalTok{(}\AttributeTok{h =} \DecValTok{0}\NormalTok{, }\AttributeTok{lty =} \DecValTok{2}\NormalTok{)}
\end{Highlighting}
\end{Shaded}

\includegraphics[width=0.9\linewidth]{07-interactions-diagnostics_files/figure-beamer/unnamed-chunk-9-1}
\end{frame}

\begin{frame}[fragile]{Continuous example: lower n \& effect size}
\phantomsection\label{continuous-example-lower-n-effect-size}
\begin{Shaded}
\begin{Highlighting}[]
\FunctionTok{set.seed}\NormalTok{(}\DecValTok{42}\NormalTok{)}
\NormalTok{n  }\OtherTok{=} \DecValTok{30}
\NormalTok{x1 }\OtherTok{=} \FunctionTok{rnorm}\NormalTok{(n)}
\NormalTok{x2 }\OtherTok{=} \FunctionTok{rnorm}\NormalTok{(n)}
\NormalTok{X  }\OtherTok{=} \FunctionTok{cbind}\NormalTok{(}\DecValTok{1}\NormalTok{, x1, x2, x1}\SpecialCharTok{*}\NormalTok{x2)}
\NormalTok{b  }\OtherTok{=} \FunctionTok{c}\NormalTok{(}\DecValTok{1}\NormalTok{, }\FloatTok{0.1}\NormalTok{, }\DecValTok{1}\NormalTok{, }\SpecialCharTok{{-}}\FloatTok{0.2}\NormalTok{)}
\NormalTok{e  }\OtherTok{=} \FunctionTok{rnorm}\NormalTok{(n)}
\NormalTok{y  }\OtherTok{=} \FunctionTok{c}\NormalTok{(X }\SpecialCharTok{\%*\%}\NormalTok{ b }\SpecialCharTok{+}\NormalTok{ e)}
\NormalTok{m  }\OtherTok{=} \FunctionTok{lm}\NormalTok{(y }\SpecialCharTok{\textasciitilde{}}\NormalTok{ x1}\SpecialCharTok{*}\NormalTok{x2)}
\end{Highlighting}
\end{Shaded}
\end{frame}

\begin{frame}[fragile]{Continuous example: lower n \& effect size}
\phantomsection\label{continuous-example-lower-n-effect-size-1}
\begin{Shaded}
\begin{Highlighting}[]
\NormalTok{beta\_hat }\OtherTok{=} \FunctionTok{coef}\NormalTok{(m); CIs }\OtherTok{=} \FunctionTok{confint}\NormalTok{(m)}
\FunctionTok{round}\NormalTok{(}\FunctionTok{cbind}\NormalTok{(beta\_hat, CIs), }\DecValTok{2}\NormalTok{)}
\end{Highlighting}
\end{Shaded}

\begin{verbatim}
##             beta_hat 2.5 % 97.5 %
## (Intercept)     1.13  0.84   1.42
## x1             -0.03 -0.26   0.21
## x2              0.83  0.54   1.11
## x1:x2          -0.38 -0.61  -0.16
\end{verbatim}

\pause

\begin{Shaded}
\begin{Highlighting}[]
\NormalTok{MEs }\OtherTok{=} \FunctionTok{margins}\NormalTok{(m, }\AttributeTok{variables =} \StringTok{"x1"}\NormalTok{, }\AttributeTok{at =} \FunctionTok{list}\NormalTok{(}\AttributeTok{x2 =} \FunctionTok{mean}\NormalTok{(x2)))}
\FunctionTok{print}\NormalTok{(}\FunctionTok{summary}\NormalTok{(MEs), }\AttributeTok{digits =} \DecValTok{3}\NormalTok{)}
\end{Highlighting}
\end{Shaded}

\begin{verbatim}
##  factor     x2   AME    SE     z     p  lower upper
##      x1 -0.122 0.021 0.113 0.185 0.853 -0.201 0.243
\end{verbatim}
\end{frame}

\begin{frame}{``The Checklist''}
\phantomsection\label{the-checklist}
\pause

\href{https://www.jstor.org/stable/25791835}{(Brambor, Clark, and Golder 2006)}:

\pause

\begin{enumerate}[<+->]
\tightlist
\item
  Include interaction terms
\item
  Include all constitutive terms
\item
  Constitutive terms are \textbf{not} unconditional marginal effects!
\item
  Calculate substantively meaningful marginal effects and standard errors
\end{enumerate}
\end{frame}

\begin{frame}{Marginal effect quantities of interest}
\phantomsection\label{marginal-effect-quantities-of-interest}
\pause

\begin{itemize}[<+->]
\tightlist
\item
  (Sample) Average Marginal Effect ((S)AME): Marginal effect for every observation, averaged
\item
  Marginal effect at the mean: Marginal effect at the mean value of the predictor

  \begin{itemize}[<+->]
  \tightlist
  \item
    Sometimes done instead at the median
  \end{itemize}
\item
  Marginal effect at representative values

  \begin{itemize}[<+->]
  \tightlist
  \item
    Calculated at values of interest for your study
  \item
    Researchers often use mean \(\pm\) one standard deviation
  \item
    But different values could be important for your study, substantive context and theory matters
  \end{itemize}
\end{itemize}
\end{frame}

\begin{frame}[fragile]{Continuous example (still with lower n \& effect)}
\phantomsection\label{continuous-example-still-with-lower-n-effect}
Consider the effect at one standard deviation above and below the mean of \texttt{x2}
(i.e.~when \texttt{x2} is -1.17 \& 0.93)

\pause

\begin{Shaded}
\begin{Highlighting}[]
\NormalTok{mx   }\OtherTok{=} \FunctionTok{mean}\NormalTok{(x2)}
\NormalTok{sdx  }\OtherTok{=} \FunctionTok{sd}\NormalTok{(x2)}
\NormalTok{vals }\OtherTok{=} \FunctionTok{c}\NormalTok{(mx }\SpecialCharTok{+}\NormalTok{ sdx, mx }\SpecialCharTok{{-}}\NormalTok{ sdx)}
\NormalTok{MEs  }\OtherTok{=} \FunctionTok{margins}\NormalTok{(m, }\AttributeTok{variables =} \StringTok{"x1"}\NormalTok{, }\AttributeTok{at =} \FunctionTok{list}\NormalTok{(}\AttributeTok{x2 =}\NormalTok{ vals))}
\FunctionTok{print}\NormalTok{(}\FunctionTok{summary}\NormalTok{(MEs), }\AttributeTok{digits =} \DecValTok{3}\NormalTok{)}
\end{Highlighting}
\end{Shaded}

\begin{verbatim}
##  factor     x2    AME    SE      z     p  lower  upper
##      x1 -1.172  0.424 0.171  2.475 0.013  0.088  0.759
##      x1  0.928 -0.382 0.154 -2.486 0.013 -0.683 -0.081
\end{verbatim}

(True effect at those \texttt{x2} values is 0.33 and -0.09)
\end{frame}

\begin{frame}[fragile]{Continuous example (still with lower n \& effect)}
\phantomsection\label{continuous-example-still-with-lower-n-effect-1}
\begin{Shaded}
\begin{Highlighting}[]
\NormalTok{pts }\OtherTok{=} \FunctionTok{coef}\NormalTok{(m)[}\DecValTok{2}\NormalTok{] }\SpecialCharTok{+} \FunctionTok{coef}\NormalTok{(m)[}\DecValTok{4}\NormalTok{] }\SpecialCharTok{*}\NormalTok{ x2}
\NormalTok{margins}\SpecialCharTok{::}\FunctionTok{cplot}\NormalTok{(m, }\AttributeTok{x =} \StringTok{"x2"}\NormalTok{, }\AttributeTok{dx =} \StringTok{"x1"}\NormalTok{, }\AttributeTok{what =} \StringTok{"effect"}\NormalTok{)}
\FunctionTok{abline}\NormalTok{(}\AttributeTok{h =} \DecValTok{0}\NormalTok{, }\AttributeTok{lty =} \DecValTok{2}\NormalTok{)}
\FunctionTok{points}\NormalTok{(}\AttributeTok{x =}\NormalTok{ x2, }\AttributeTok{y =}\NormalTok{ pts, }\AttributeTok{pch =} \DecValTok{19}\NormalTok{, }\AttributeTok{col =} \StringTok{"\#d55e0080"}\NormalTok{)}
\end{Highlighting}
\end{Shaded}

\includegraphics[width=0.9\linewidth]{07-interactions-diagnostics_files/figure-beamer/unnamed-chunk-14-1}
\end{frame}

\begin{frame}{Diagnostics}
\phantomsection\label{diagnostics}
\pause

Recall the assumptions we've made to enable inference for OLS:

\begin{enumerate}
\tightlist
\item
  Linear in parameters
\item
  Random sampling
\item
  No perfect collinearity: In the sample, no explanatory variable is constant, and there is no \textbf{exact} linear relationship between any explanatory variables
\item
  Zero conditional mean: The error term has the same expected value at any value of the explanatory variables: \(\ensuremath{\mathbb{E}\left[\varepsilon \mid x_1, \dots, x_k\right]} = 0\)
\item
  Homoscedasticity: The error term has the same variance at any value of the explanatory variables; \(\ensuremath{\mathop{\mathrm{Var}}\left[\varepsilon \mid x_1, \dots, x_k\right]} = \sigma^2\)
\item
  \(\varepsilon \sim \mathcal{N}\left( 0, \sigma^2 \right)\)
\end{enumerate}
\end{frame}

\begin{frame}[fragile]{Diagnostics: The Simulated Data}
\phantomsection\label{diagnostics-the-simulated-data}
\begin{Shaded}
\begin{Highlighting}[]
\FunctionTok{set.seed}\NormalTok{(}\DecValTok{42}\NormalTok{)}
\NormalTok{n  }\OtherTok{=} \DecValTok{100}
\NormalTok{x1 }\OtherTok{=} \FunctionTok{rnorm}\NormalTok{(n)}
\NormalTok{x2 }\OtherTok{=} \FunctionTok{rnorm}\NormalTok{(n)}
\NormalTok{X  }\OtherTok{=} \FunctionTok{cbind}\NormalTok{(}\DecValTok{1}\NormalTok{, x1, x2, x2}\SpecialCharTok{\^{}}\DecValTok{2}\NormalTok{)}
\NormalTok{b  }\OtherTok{=} \FunctionTok{c}\NormalTok{(}\DecValTok{1}\NormalTok{, }\DecValTok{1}\NormalTok{, }\DecValTok{1}\NormalTok{, }\SpecialCharTok{{-}}\DecValTok{2}\NormalTok{)}
\NormalTok{e  }\OtherTok{=} \FunctionTok{rnorm}\NormalTok{(n)}
\NormalTok{y  }\OtherTok{=} \FunctionTok{c}\NormalTok{(X }\SpecialCharTok{\%*\%}\NormalTok{ b }\SpecialCharTok{+}\NormalTok{ e)}
\NormalTok{m  }\OtherTok{=} \FunctionTok{lm}\NormalTok{(y }\SpecialCharTok{\textasciitilde{}}\NormalTok{ x1 }\SpecialCharTok{+}\NormalTok{ x2 }\SpecialCharTok{+} \FunctionTok{I}\NormalTok{(x2}\SpecialCharTok{\^{}}\DecValTok{2}\NormalTok{))}
\end{Highlighting}
\end{Shaded}
\end{frame}

\begin{frame}[fragile]{Diagnostics: Linearity}
\phantomsection\label{diagnostics-linearity}
\pause

\begin{Shaded}
\begin{Highlighting}[]
\FunctionTok{library}\NormalTok{(ggplot2)}
\NormalTok{d }\OtherTok{=} \FunctionTok{data.frame}\NormalTok{(}
    \AttributeTok{x =} \FunctionTok{c}\NormalTok{(x1, x2),}
    \AttributeTok{Variable =} \FunctionTok{rep}\NormalTok{(}\FunctionTok{c}\NormalTok{(}\StringTok{"x1"}\NormalTok{, }\StringTok{"x2"}\NormalTok{), }\AttributeTok{each =}\NormalTok{ n),}
    \AttributeTok{r =} \FunctionTok{rep}\NormalTok{(}\FunctionTok{residuals}\NormalTok{(m), }\DecValTok{2}\NormalTok{)}
\NormalTok{)}
\FunctionTok{ggplot}\NormalTok{(}\AttributeTok{data =}\NormalTok{ d, }\AttributeTok{mapping =} \FunctionTok{aes}\NormalTok{(}\AttributeTok{x =}\NormalTok{ x, }\AttributeTok{y =}\NormalTok{ r)) }\SpecialCharTok{+}
    \FunctionTok{facet\_wrap}\NormalTok{(}\SpecialCharTok{\textasciitilde{}}\NormalTok{Variable) }\SpecialCharTok{+}
    \FunctionTok{geom\_point}\NormalTok{() }\SpecialCharTok{+}
    \FunctionTok{geom\_smooth}\NormalTok{(}\AttributeTok{color =} \StringTok{"\#0072b2"}\NormalTok{, }\AttributeTok{fill =} \StringTok{"\#0072b280"}\NormalTok{) }\SpecialCharTok{+}
    \FunctionTok{geom\_hline}\NormalTok{(}\AttributeTok{yintercept =} \DecValTok{0}\NormalTok{, }\AttributeTok{linetype =} \StringTok{"dashed"}\NormalTok{) }\SpecialCharTok{+}
    \FunctionTok{ylab}\NormalTok{(}\StringTok{"Residual"}\NormalTok{) }\SpecialCharTok{+}
    \FunctionTok{theme\_bw}\NormalTok{() }\SpecialCharTok{+}
    \FunctionTok{theme}\NormalTok{(}\AttributeTok{axis.title.x =} \FunctionTok{element\_blank}\NormalTok{())}
\end{Highlighting}
\end{Shaded}
\end{frame}

\begin{frame}{Diagnostics: Linearity}
\phantomsection\label{diagnostics-linearity-1}
\includegraphics[width=0.9\linewidth]{07-interactions-diagnostics_files/figure-beamer/unnamed-chunk-16-1}
\end{frame}

\begin{frame}[fragile]{Diagnostics: Linearity}
\phantomsection\label{diagnostics-linearity-2}
\begin{Shaded}
\begin{Highlighting}[]
\FunctionTok{plot}\NormalTok{(}\AttributeTok{x =} \FunctionTok{fitted}\NormalTok{(m), }\AttributeTok{y =} \FunctionTok{residuals}\NormalTok{(m))}
\FunctionTok{abline}\NormalTok{(}\AttributeTok{h =} \DecValTok{0}\NormalTok{, }\AttributeTok{lty =} \DecValTok{2}\NormalTok{)}
\end{Highlighting}
\end{Shaded}

\includegraphics[width=0.9\linewidth]{07-interactions-diagnostics_files/figure-beamer/unnamed-chunk-17-1}
\end{frame}

\begin{frame}[fragile]{Diagnostics: Multicollinearity}
\phantomsection\label{diagnostics-multicollinearity}
\pause

\begin{itemize}[<+->]
\tightlist
\item
  Under perfect multicollinearity, the OLS estimator is undefined
\item
  But even imperfect multicollinearity increases variance
\item
  We can assess multicollinearity using variance inflation factors (VIF)

  \begin{itemize}[<+->]
  \tightlist
  \item
    Ratio of coefficient variance in the full model vs just using that predictor
  \item
    Square root gives standard error increase relative to 0 correlation to other predictors
  \end{itemize}
\item
  \emph{By convention}, VIF \textgreater{} 10 or even VIF \textgreater{} 5 indicates a problem
\end{itemize}

\pause

\begin{Shaded}
\begin{Highlighting}[]
\NormalTok{car}\SpecialCharTok{::}\FunctionTok{vif}\NormalTok{(m)}
\end{Highlighting}
\end{Shaded}

\begin{verbatim}
##       x1       x2  I(x2^2) 
## 1.059565 1.009498 1.065707
\end{verbatim}
\end{frame}

\begin{frame}[fragile]{Diagnostics: Independence of errors}
\phantomsection\label{diagnostics-independence-of-errors}
\pause

\begin{itemize}[<+->]
\tightlist
\item
  No statistical test for independence from the predictors
\item
  Durbin-Watson test statistic tests for autocorrelation
\end{itemize}

\pause

~

\small

\begin{Shaded}
\begin{Highlighting}[]
\NormalTok{lmtest}\SpecialCharTok{::}\FunctionTok{dwtest}\NormalTok{(m)}
\end{Highlighting}
\end{Shaded}

\begin{verbatim}
## 
##  Durbin-Watson test
## 
## data:  m
## DW = 1.9814, p-value = 0.4637
## alternative hypothesis: true autocorrelation is greater than 0
\end{verbatim}

\normalsize
\end{frame}

\begin{frame}[fragile]{Diagnostics: Homoscedasticity}
\phantomsection\label{diagnostics-homoscedasticity}
\begin{Shaded}
\begin{Highlighting}[]
\NormalTok{r }\OtherTok{=} \FunctionTok{residuals}\NormalTok{(m); f }\OtherTok{=} \FunctionTok{fitted}\NormalTok{(m); l }\OtherTok{=} \FunctionTok{loess}\NormalTok{(r }\SpecialCharTok{\textasciitilde{}}\NormalTok{ f)}
\FunctionTok{plot}\NormalTok{(}\AttributeTok{x =}\NormalTok{ f, }\AttributeTok{y =}\NormalTok{ r); }\FunctionTok{abline}\NormalTok{(}\AttributeTok{h =} \DecValTok{0}\NormalTok{, }\AttributeTok{lty =} \DecValTok{2}\NormalTok{)}
\FunctionTok{lines}\NormalTok{(}\AttributeTok{x =}\NormalTok{ f[}\FunctionTok{order}\NormalTok{(f)], }\AttributeTok{y =} \FunctionTok{predict}\NormalTok{(l)[}\FunctionTok{order}\NormalTok{(f)], }\AttributeTok{col =} \StringTok{"red"}\NormalTok{)}
\end{Highlighting}
\end{Shaded}

\includegraphics[width=0.9\linewidth]{07-interactions-diagnostics_files/figure-beamer/unnamed-chunk-20-1}
\end{frame}

\begin{frame}[fragile]{Diagnostics: Homoscedasticity}
\phantomsection\label{diagnostics-homoscedasticity-1}
\begin{Shaded}
\begin{Highlighting}[]
\FunctionTok{plot}\NormalTok{(}\FunctionTok{residuals}\NormalTok{(m))}
\FunctionTok{abline}\NormalTok{(}\AttributeTok{h =} \DecValTok{0}\NormalTok{, }\AttributeTok{lty =} \DecValTok{2}\NormalTok{)}
\end{Highlighting}
\end{Shaded}

\includegraphics[width=0.9\linewidth]{07-interactions-diagnostics_files/figure-beamer/unnamed-chunk-21-1}
\end{frame}

\begin{frame}[fragile]{Diagnostics: Homoscedasticity}
\phantomsection\label{diagnostics-homoscedasticity-2}
\begin{Shaded}
\begin{Highlighting}[]
\NormalTok{m2 }\OtherTok{=} \FunctionTok{lm}\NormalTok{(y }\SpecialCharTok{\textasciitilde{}}\NormalTok{ x1 }\SpecialCharTok{+}\NormalTok{ x2)}
\NormalTok{r }\OtherTok{=} \FunctionTok{residuals}\NormalTok{(m2); f }\OtherTok{=} \FunctionTok{fitted}\NormalTok{(m2); l }\OtherTok{=} \FunctionTok{loess}\NormalTok{(r }\SpecialCharTok{\textasciitilde{}}\NormalTok{ f)}
\FunctionTok{plot}\NormalTok{(}\AttributeTok{x =}\NormalTok{ f, }\AttributeTok{y =}\NormalTok{ r); }\FunctionTok{abline}\NormalTok{(}\AttributeTok{h =} \DecValTok{0}\NormalTok{, }\AttributeTok{lty =} \DecValTok{2}\NormalTok{)}
\FunctionTok{lines}\NormalTok{(}\AttributeTok{x =}\NormalTok{ f[}\FunctionTok{order}\NormalTok{(f)], }\AttributeTok{y =} \FunctionTok{predict}\NormalTok{(l)[}\FunctionTok{order}\NormalTok{(f)], }\AttributeTok{col =} \StringTok{"red"}\NormalTok{)}
\end{Highlighting}
\end{Shaded}

\includegraphics[width=0.9\linewidth]{07-interactions-diagnostics_files/figure-beamer/unnamed-chunk-22-1}
\end{frame}

\begin{frame}[fragile]{Diagnostics: Homoscedasticity}
\phantomsection\label{diagnostics-homoscedasticity-3}
\begin{Shaded}
\begin{Highlighting}[]
\FunctionTok{plot}\NormalTok{(}\FunctionTok{residuals}\NormalTok{(m2))}
\FunctionTok{abline}\NormalTok{(}\AttributeTok{h =} \DecValTok{0}\NormalTok{, }\AttributeTok{lty =} \DecValTok{2}\NormalTok{)}
\end{Highlighting}
\end{Shaded}

\includegraphics[width=0.9\linewidth]{07-interactions-diagnostics_files/figure-beamer/unnamed-chunk-23-1}
\end{frame}

\begin{frame}{Diagnostics: Homoscedasticity}
\phantomsection\label{diagnostics-homoscedasticity-4}
\begin{itemize}[<+->]
\tightlist
\item
  One option is the Breusch--Pagan test (Breusch and Pagan 1979)

  \begin{itemize}[<+->]
  \tightlist
  \item
    Regress \(g = \hat{\varepsilon}^2 / \hat{\sigma}^2\) on the predictors
  \item
    Then the test statistic \((SST - SSR) / 2\) is distributed \(\chi_k^2\)
  \item
    Tests for \textbf{linear} heteroscedasticity
  \end{itemize}
\item
  Another is the White test (White 1980)

  \begin{itemize}[<+->]
  \tightlist
  \item
    Regress \(\hat{\varepsilon}^2\) on the predictors, their squares, and interactions
  \item
    Then the test statistic \(nR^2\) is distributed \(\chi_{p-1}^2\)
  \item
    More general but loses power with many regressors
  \end{itemize}
\item
  Unfortunately cannot distinguish between heteroscedasticity \& misspecification
\end{itemize}
\end{frame}

\begin{frame}[fragile]{Diagnostics: Homoscedasticity}
\phantomsection\label{diagnostics-homoscedasticity-5}
\begin{Shaded}
\begin{Highlighting}[]
\NormalTok{lmtest}\SpecialCharTok{::}\FunctionTok{bptest}\NormalTok{(m)}
\end{Highlighting}
\end{Shaded}

\begin{verbatim}
## 
##  studentized Breusch-Pagan test
## 
## data:  m
## BP = 8.5267, df = 3, p-value = 0.03629
\end{verbatim}

\begin{Shaded}
\begin{Highlighting}[]
\NormalTok{skedastic}\SpecialCharTok{::}\FunctionTok{white}\NormalTok{(m, }\AttributeTok{interactions =} \ConstantTok{TRUE}\NormalTok{)}
\end{Highlighting}
\end{Shaded}

\begin{verbatim}
## # A tibble: 1 x 5
##   statistic p.value parameter method       alternative
##       <dbl>   <dbl>     <dbl> <chr>        <chr>      
## 1      12.1   0.205         9 White's Test greater
\end{verbatim}
\end{frame}

\begin{frame}[fragile]{Diagnostics: Normality of errors}
\phantomsection\label{diagnostics-normality-of-errors}
\pause

\begin{itemize}[<+->]
\tightlist
\item
  Shapiro-Wilk test
\end{itemize}

\begin{Shaded}
\begin{Highlighting}[]
\FunctionTok{shapiro.test}\NormalTok{(}\FunctionTok{residuals}\NormalTok{(m))}
\end{Highlighting}
\end{Shaded}

\begin{verbatim}
## 
##  Shapiro-Wilk normality test
## 
## data:  residuals(m)
## W = 0.98908, p-value = 0.5906
\end{verbatim}

\begin{itemize}[<+->]
\tightlist
\item
  (See also the Kolmogorov-Smirnov, Jarque-Bera, and Anderson-Darling tests)
\end{itemize}
\end{frame}

\begin{frame}[fragile]{Diagnostics: Normality of errors}
\phantomsection\label{diagnostics-normality-of-errors-1}
\begin{itemize}[<+->]
\tightlist
\item
  Q-Q plot
\end{itemize}

\begin{Shaded}
\begin{Highlighting}[]
\FunctionTok{qqnorm}\NormalTok{(}\FunctionTok{residuals}\NormalTok{(m))}
\end{Highlighting}
\end{Shaded}

\includegraphics[width=0.9\linewidth]{07-interactions-diagnostics_files/figure-beamer/unnamed-chunk-26-1}
\end{frame}

\begin{frame}{Solutions}
\phantomsection\label{solutions}
\pause

\begin{enumerate}[<+->]
\tightlist
\item
  Change the OLS model specification

  \begin{enumerate}[<+->]
  [a.]
  \tightlist
  \item
    Transform the predictor(s) and/or the outcomes?
  \item
    Add interaction terms?
  \item
    Are there omitted variables?
  \end{enumerate}
\item
  Models for correlated errors (more on this after the midterm)
\item
  Look for ``influential'' observations

  \begin{enumerate}[<+->]
  [a.]
  \tightlist
  \item
    DFBETAS, Cook's D
  \item
    \emph{Should} we delete outliers??
  \end{enumerate}
\end{enumerate}
\end{frame}

\begin{frame}[fragile]{Influential observations}
\phantomsection\label{influential-observations}
\pause

\begin{Shaded}
\begin{Highlighting}[]
\FunctionTok{set.seed}\NormalTok{(}\DecValTok{123}\NormalTok{)}
\NormalTok{x }\OtherTok{=} \FunctionTok{rnorm}\NormalTok{(}\DecValTok{30}\NormalTok{); y }\OtherTok{=}\NormalTok{ x }\SpecialCharTok{+} \FunctionTok{rnorm}\NormalTok{(}\DecValTok{30}\NormalTok{)}
\NormalTok{m }\OtherTok{=} \FunctionTok{lm}\NormalTok{(y }\SpecialCharTok{\textasciitilde{}}\NormalTok{ x)}
\NormalTok{x1 }\OtherTok{=} \FunctionTok{c}\NormalTok{(x, }\DecValTok{7}\NormalTok{); y1 }\OtherTok{=} \FunctionTok{c}\NormalTok{(y, }\SpecialCharTok{{-}}\DecValTok{2}\NormalTok{)}
\NormalTok{m1 }\OtherTok{=} \FunctionTok{lm}\NormalTok{(y1 }\SpecialCharTok{\textasciitilde{}}\NormalTok{ x1)}
\end{Highlighting}
\end{Shaded}
\end{frame}

\begin{frame}[fragile]{Influential observations}
\phantomsection\label{influential-observations-1}
\begin{Shaded}
\begin{Highlighting}[]
\NormalTok{point\_color }\OtherTok{=} \StringTok{"\#80808080"}
\FunctionTok{plot}\NormalTok{(}
\NormalTok{    x, y, }\AttributeTok{pch =} \DecValTok{19}\NormalTok{, }\AttributeTok{col =}\NormalTok{ point\_color,}
    \AttributeTok{xlim =} \FunctionTok{c}\NormalTok{(}\SpecialCharTok{{-}}\DecValTok{3}\NormalTok{, }\DecValTok{8}\NormalTok{), }\AttributeTok{ylim =} \FunctionTok{c}\NormalTok{(}\SpecialCharTok{{-}}\DecValTok{3}\NormalTok{, }\DecValTok{5}\NormalTok{)}
\NormalTok{)}
\FunctionTok{abline}\NormalTok{(m)}
\end{Highlighting}
\end{Shaded}

\begin{center}\includegraphics[width=0.8\linewidth]{07-interactions-diagnostics_files/figure-beamer/unnamed-chunk-28-1} \end{center}
\end{frame}

\begin{frame}[fragile]{Influential observations}
\phantomsection\label{influential-observations-2}
\begin{Shaded}
\begin{Highlighting}[]
\NormalTok{point\_color }\OtherTok{=} \FunctionTok{c}\NormalTok{(}\FunctionTok{rep}\NormalTok{(}\StringTok{"\#80808080"}\NormalTok{, }\DecValTok{30}\NormalTok{), }\StringTok{"\#bf5700"}\NormalTok{)}
\FunctionTok{plot}\NormalTok{(}
\NormalTok{    x1, y1, }\AttributeTok{pch =} \DecValTok{19}\NormalTok{, }\AttributeTok{col =}\NormalTok{ point\_color,}
    \AttributeTok{xlim =} \FunctionTok{c}\NormalTok{(}\SpecialCharTok{{-}}\DecValTok{3}\NormalTok{, }\DecValTok{8}\NormalTok{), }\AttributeTok{ylim =} \FunctionTok{c}\NormalTok{(}\SpecialCharTok{{-}}\DecValTok{3}\NormalTok{, }\DecValTok{5}\NormalTok{)}
\NormalTok{)}
\FunctionTok{abline}\NormalTok{(m1)}
\end{Highlighting}
\end{Shaded}

\begin{center}\includegraphics[width=0.8\linewidth]{07-interactions-diagnostics_files/figure-beamer/unnamed-chunk-29-1} \end{center}
\end{frame}

\begin{frame}[fragile]{Methods to detect influential observations}
\phantomsection\label{methods-to-detect-influential-observations}
\begin{itemize}[<+->]
\tightlist
\item
  DFBETA: Coefficient change when you omit an observation

  \begin{itemize}[<+->]
  \tightlist
  \item
    DFBETA\(_{ij} = \hat{\beta}_j - \hat{\beta}_{(i)j}\)
  \item
    \(\hat{\beta}_j\) is the coefficient estimate using all observations
  \item
    \(\hat{\beta}_{(i)j}\) is the estimate omitting the \(i\)th observation
  \item
    Often you really only care about the absolute value
  \end{itemize}
\item
  DFBETA\textbf{S}: standardized DFBETA

  \begin{itemize}[<+->]
  \tightlist
  \item
    DFBETAS\(_{ij} = \frac{\hat{\beta}_j - \hat{\beta}_{(i)j}}{\mathop{\mathrm{se}}\left(\hat{\beta}_j\right)}\); look at \(|\)DFBETAS\(| > 2/\sqrt{n}\)
  \end{itemize}
\item
  Cook's \(D\): All parameters' change when you omit an observation

  \begin{itemize}[<+->]
  \tightlist
  \item
    Hat matrix \(H = X (X'X)^{-1} X'\)
  \item
    \(D_i = \frac{e_i^2}{ps^2} \left(\frac{h_{ii}}{(1 - h_{ii})^2}\right)\); look at \(D_i > 1\)
  \item
    \(e_i\) is residual \(i\), \(p\) is model parameters, \(s^2\) is mean squared error
  \end{itemize}
\item
  R functions: \texttt{dfbeta()}, \texttt{dfbetas()}, \texttt{cooks.distance()}, \texttt{influence.measures()}
\end{itemize}
\end{frame}

\end{document}
